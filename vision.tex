% +++
% latex="lualatex"
% +++
\documentclass[
	fontsize=9pt,
	twocolumn,
	hanging_punctuation,
	paper=a4paper,
	gutter=15mm,
	fore-edge=15mm,
	head_space=15mm,
	foot_space=15mm,
]{jlreq}
\usepackage[no-math,match,deluxe]{luatexja-preset}

% \usepackage[paper=a4paper,margin=15mm]{geometry}

% \usepackage{textcomp}
% \usepackage{luatexja-otf}

\usepackage{graphicx,xcolor}
\usepackage{hyperref}
\usepackage{pxrubrica}
\usepackage{tcolorbox}
\usepackage{bxwareki}
\usepackage{indentfirst}

\setmainfont{Exo 2}
\setsansfont{Exo 2}
\setmainjfont{FOT-MatissePro-DB}[
	AltFont={
		{
			Range={
				"4E00-"9FFF, % CJK 統合漢字
				"3400-"4DFF, % CJK 統合漢字拡張 A
				"20000-"2EBE0, % CJK 統合漢字拡張 B-F
				"2460-"24FF, % 囲み英数字
				"3200-"32FF, % 囲み CJK 文字・月
				"1F100-"1F2FF % 囲み英数字補助、漢字補助
			},
			Font=FOT-RodinNTLGPro-DB,
		},
	},
	BoldFeatures={
		Font=FOT-MatissePro-EB,
		AltFont={
			 {
				Range={
					"4E00-"9FFF, % CJK 統合漢字
					"3400-"4DFF, % CJK 統合漢字拡張 A
					"20000-"2EBE0, % CJK 統合漢字拡張 B-F
					"2460-"24FF, % 囲み英数字
					"3200-"32FF, % 囲み CJK 文字・月
					"1F100-"1F2FF % 囲み英数字補助、漢字補助
				},
				Font=FOT-RodinNTLGPro-EB,
			},
		},
	},
	YokoFeatures={JFM=jlreq},   % jlreqのJFMを維持する
	TateFeatures={JFM=jlreqv},  % https://qiita.com/zr_tex8r/items/91ae1dcc9c3afce7fa8c
]
\setsansjfont{FOT-RodinNTLGPro-DB}[
	AltFont={
		{
			Range={
				"4E00-"9FFF, % CJK 統合漢字
				"3400-"4DFF, % CJK 統合漢字拡張 A
				"20000-"2EBE0, % CJK 統合漢字拡張 B-F
				"2460-"24FF, % 囲み英数字
				"3200-"32FF, % 囲み CJK 文字・月
				"1F100-"1F2FF % 囲み英数字補助、漢字補助
			},
			Font=FOT-RodinNTLGPro-DB,
		},
	},
	BoldFeatures={
		Font=FOT-MatissePro-EB,
		AltFont={
			 {
				Range={
					"4E00-"9FFF, % CJK 統合漢字
					"3400-"4DFF, % CJK 統合漢字拡張 A
					"20000-"2EBE0, % CJK 統合漢字拡張 B-F
					"2460-"24FF, % 囲み英数字
					"3200-"32FF, % 囲み CJK 文字・月
					"1F100-"1F2FF % 囲み英数字補助、漢字補助
				},
				Font=FOT-RodinNTLGPro-EB,
			},
		},
	},
	YokoFeatures={JFM=jlreq},   % jlreqのJFMを維持する
	TateFeatures={JFM=jlreqv},  % https://qiita.com/zr_tex8r/items/91ae1dcc9c3afce7fa8c
]

\setlength{\parindent}{0pt}
\newcommand{\↴}{{\jfontspec{nishiki-teki}↴}}

\title{VISION 戦術・特殊効果表}
\author{ひとみさん}
\date{\warekitoday}


\begin{document}
\maketitle

\section*{速攻}
(自動β)〔このキャラクター〕がプレイされて場に出た場合、このキャラクターをアクティブ状態にする。

\section*{奇襲}
(自動β)〔手札にあるこのキャラクターカード〕はコマンドカードのタイミングでプレイ出来る。
更に、このカードのプレイは「キャラクターカードは1ターンに1枚しかプレイ出来ない」という制限に含まれない。

(自動β)〔手札にあるこのキャラクターカード〕が「速攻」を持ち、相手キャラクターの攻撃に対する
直接の干渉でプレイされる場合、プレイ時に、場に出るキャラクターにその攻撃を防御させる事を選択出来る。

\section*{先制}
(自動α)〔このキャラクター〕の戦闘ダメージは「先制のタイミング」で解決される。

\section*{貫通}
(自動γ)攻撃を行っているこのキャラクターが相手キャラクターに戦闘ダメージを与えた場合、
相手プレイヤーにXダメージを与える。Xはこのキャラクターの攻撃力から相手キャラクターの耐久力と
蓄積ダメージを引いた値に等しい。但し、Xが0以下になる場合、この効果は無効になる。
また、この効果ではグレイズは発生しない。この効果は重複しない。

\section*{隠密}
(自動γ)このキャラクターが攻撃を行った場合、戦闘開始時にこのキャラクターと戦闘している
相手キャラクターの防御を取り消す。但し、防御を行った相手キャラクターがこのキャラクターと
同じ種族を持つ場合、この効果は無効になる。

\pagebreak
\section*{警戒}
(自動γ)相手プレイヤーのアクティブフェイズ開始時、〔このキャラクター〕をアクティブ状態にする。

\section*{即死}
(自動γ)このキャラクターが相手キャラクターに戦闘ダメージを与えた場合、戦闘ダメージを
受けた相手キャラクターを決死状態にする。

\section*{人形}
(自動α)〔このキャラクター〕は攻撃を行う事が出来ない。

(自動β)〔このカード〕のプレイは「キャラクターカードは1ターンに1枚しかプレイ出来ない」
という制限に含まれない。このキャラクターカードをプレイしたターン、「人形」を持つ
キャラクターカードをプレイする事は出来ない。

\section*{変身}
(自動γ)〔このキャラクター〕が攻撃、または防御を行う場合、 同時に次のあなたのターン開始時まで
変身状態にしても良い。

(常時)0: この効果の解決時にこのキャラクターがアクティブ状態の場合、次のあなたのターン開始時まで
〔このキャラクター〕を変身状態にする。

\section*{封術}
(自動α)〔このキャラクター〕はスペルカードの術者として扱う事が出来ない。
但し、あなたのノードが6枚以上の場合、この効果は無効になる。

\section*{伝説}
特に効果はありません。

\pagebreak
\section*{耐性 A}
(自動α)〔このキャラクター〕はAの種族を持つキャラクターから戦闘ダメージを受けない。

\section*{マナチャージ(X)}
(自分ターン)\↴:〔あなたのデッキの上のカードX枚〕をスリープ状態でノードに加える。

\section*{加護(X)}
(自動γ)〔このキャラクター〕が相手プレイヤーのコマンドカード、及びキャラクター効果の
目標になった場合、相手プレイヤーはⓍ支払っても良い。支払わない場合、その相手プレイヤーのカードの
効果を無効とし、破棄する。

\section*{装備}
〔このカード〕を、目標の〔あなたの場のキャラクター1枚〕にセットする。

■「装備」はキャラクターにセットするカードである事を表します。
キャラクターには「装備」を持つカードは1枚しかセット出来ないという制限がある為、
同じキャラクターに複数枚の「装備」を持つカードをセットする事は出来ません。
何らかの理由により2枚以上セットされた場合、直ちに1枚になる様に選んで冥界に置きます。
「装備」を持つカードの事を、ゲーム内では「装備カード」と呼ぶ場合があります。

\section*{呪符}
〔このカード〕を、目標の〔キャラクター1枚〕にセットする。

■「呪符」も「装備」と同様に、キャラクターにセットするカードである事を表します。
「装備」と違う点は、相手キャラクターにもセット出来るという事、そしてセット出来る枚数の
制限が無いという事です。「呪符」を持つカードの事を、ゲーム内では「呪符カード」と
呼ぶ場合があります。

\section*{世界呪符}
〔このカード〕を、あなたの場にアクティブ状態でセットする。
〔あなた〕はこのターン「世界呪符」を持つ他のスペルカードをプレイする事は出来ない。

■「世界呪符」も「装備/場」と同様に、自分の場にセットするカードである事を表します。
「装備/場」と違う点は、セット出来る枚数の制限が無いという事です。その代わり、1ターン中には
1枚しかプレイする事が出来ません。

\pagebreak
\section*{装備/場}
〔このカード〕を、あなたの場にアクティブ状態でセットする。

■「装備/場」を持つカードは自分の場にセットするカードである事を表します。
「装備/場」を持つカードはお互いのプレイヤーの場に、それぞれ1枚ずつしかセットする事が出来ません。

\section*{幻想生物}
〔このカード〕を、あなたの場にスリープ状態でセットする。このカードは場にセットされた場合、
以降、キャラクターとして扱う。

■「幻想生物」は場に出た後はキャラクターとして扱われます。

\section*{維持コスト(X)}
(自動γ)あなたのメンテナンスフェイズの規定の効果の解決時に〔あなた〕はⓍ支払っても良い。
支払わない場合、〔このカード〕を破棄する。但し、あなたの場に、名称にこのカードの術者が
含まれているキャラクターがいる場合、この効果を無視する事が出来る。

\section*{壱符}
(自動γ)〔このキャラクター〕が決死状態になった場合、
〔あなたのサイドボードにある、名称に「A」を含み「弐符」を持つキャラクターカード1枚〕を、
あなたの場にスリープ状態で出しても良い。場に出した場合、〔このキャラクター〕にセットされている
カード全てをそのキャラクターに移しても良い。その後、〔このキャラクター〕を破棄する。
Aはこのカードの名称である「符ノ壱``B"」のBに等しい。

\section*{弐符}
(自動β)〔このキャラクター〕が「壱符」の効果以外で場に出た場合、
このキャラクターをゲームから除外する。この効果は他の場に出た場合の効果よりも先に解決する。

■「壱符」の効果で「弐符」を持つキャラクターを場に出す事を「スペルブレイク」と呼びます。

\pagebreak
\section*{連結}
(自動α)〔このキャラクター、または、キャラクターではないこのカード〕の名称は「A」、
「B」、および、「C」、…、としても扱う。但し、「A」、「B」、「C」、…、がカードの名称ではなく、
「~X枚」のように条件で指定される場合は名称として扱わない。

(自動β)〔このカード〕のプレイが解決される場合、または、プレイされずに場に出る場合、
〔あなたの場、手札、冥界のいずれかにある、名称に「A」を含むカード、「B」を含むカード、
および、名称に「C」を含むカード、…、それぞれ1枚〕をゲームから除外しても良い。
除外しない場合、このカードをゲームから除外する。但し、「A」、「B」、「C」、…、がカードの
名称ではなく、「~X 枚」のように条件で指定される場合はその条件に従う。

■「連結」は、複数のカードを除外する事でより強力なカードを使用する効果です。

\section*{ラストスペル}
(自動β)あなたの場にこのカードの術者がいる場合、コマンドカードのタイミングでプレイしても良い。

■「ラストスペル」を持つカード(黒いカード)は術者がいる場合はコマンドカードのタイミングで
プレイが可能です。

\section*{ラストワード}
(自動β)〔このカード〕はあなたの場にこのカードの術者がいない場合はプレイ出来ない。

(自動β)〔このカード〕がプレイされずに場に出る場合、このカードの術者のいない場に出る事は
出来ず、破棄される。

(自動β)〔プレイされているこのカード〕が他のプレイされているカードの目標になった場合、
そのカードのプレイを無効とし、破棄する。

■「ラストワード」(赤いカード)は術者がいる場合のみプレイ出来る、強力なスペルカードです。
基本的にプレイを無効とすることができません。

\section*{オーバードライブ}
(自動β)〔このカード〕のプレイが解決される場合、または、プレイされずに場に出る場合、
〔あなたの冥界にある1枚以上の任意の枚数の、このカードの術者を名称に含むキャラクターカード、
または術者がこのカードと同じスペルカード〕をゲームから除外しても良い。除外しない場合、
このカードをゲームから除外する。

■「オーバードライブ」(紫のカード)は冥界にあるカードを除外する事で強力なカードを使用する効果です。

\pagebreak
\section*{神器}
(自動β)〔このカード〕は、「伝説」を持つキャラクターにしかセットすることが出来ない。
また、〔このカード〕をプレイする場合、「伝説」を持たないキャラクターを目標にする事は出来ない。

「神器」は「装備」に付属する事があります。それらのカードは「伝説」を持たないキャラクターに
セットする事が出来ません。

\section*{リーダー}
(自動α)〔このキャラクター〕はコマンドカードの効果の目標にならず、
セットされている「人気」の枚数により、対応した効果を得る。

(自動β)〔このキャラクター〕がプレイされて場に出た場合、
あなたのデッキの上のカード1枚を〔このキャラクター〕にセットする。
そのカードは以後、「人気」として扱う。

(自動γ)あなたのターン中に相手プレイヤーがダメージを受けた場合、
あなたのデッキの上のカード1枚を〔このキャラクター〕にセットする。
そのカードは以後、「人気」として扱う。この効果は1ターンに1度しか解決されない。

(自動γ)ターン終了時、このキャラクターに「人気」が4枚以上セットされている場合、
1枚になるように破棄する。

(相手ターン)0:〔このキャラクターにセットされている「人気」1枚〕を破棄する。
このターン、〔あなた〕が次にダメージを受ける場合、そのダメージを無効とする。


■「リーダー」を持つカードはお互いのプレイヤーの場に、それぞれ1枚ずつしかセットする事が出来ません。

\section*{抵抗(X)}
(自動β)〔プレイされているこのカード〕が〔相手プレイヤーのカード〕の効果の目標になった場合、
相手プレイヤーはⓍ支払っても良い。支払わない場合、その相手プレイヤーのカードの効果を無効とし、破棄する。

■「抵抗(X)」を持つカードのプレイを無効にする場合、通常のコストに加えて追加のコストを
支払う必要があります。キャラクターカード等の場合、場に出た後は何の効果もありません。

\section*{ターン1枚制限}
(自動β)〔このカード〕をプレイした場合、またはプレイしたものとして解決した場合、
ターン終了時まで〔あなた〕はこのカードと同名のカードをプレイする事は出来ない。但し、
このカードのプレイが無効になった場合、この効果は無効になる。

\clearpage
\appendix\onecolumn\small
\section{付録: ターン進行の流れ}
\subsection{ターン開始時}
\begin{enumerate}
	\item 全てのキャラクターに蓄積されているダメージをリセットする。
	\item ターンの間が条件の効果の適用を開始する。
	\item ターン開始時がタイミングの効果を解決する。
		この時、ターン開始時が適用のタイミングである「自動γ」も解決する。
	\item アクティブフェイズへ移行する。
\end{enumerate}
\subsection{アクティブフェイズ}
\begin{description}
	\item[開始時]
	\begin{enumerate}
		\item ターンプレイヤーの場のカード全てアクティブ状態にする。
			これはアクティブフェイズの規定の効果である。
		\item アクティブフェイズ開始時がタイミングの効果を解決する。
			この時、アクティブフェイズ開始時が適用のタイミングである「自動γ」も解決する。
		\item ターンプレイヤーが優先権を得る。
	\end{enumerate}
	\item[終了時]
	\begin{enumerate}
		\item アクティブフェイズ終了時がタイミングの効果を解決する。
			この時、アクティブフェイズ終了時が適用タイミングである「自動γ」も解決する。
		\item メンテナンスフェイズに移行する。
	\end{enumerate}
\end{description}
\subsection{メンテナンスフェイズ}
\begin{description}
	\item[開始時]
		\begin{enumerate}
			\item ターンプレイヤーの場に「維持コスト」を持つカードが存在する場合、
				「維持コスト」の効果を解決する。
				これはメンテナンスフェイズの規定の効果である。
			\item メンテナンスフェイズ開始時がタイミングの効果を解決する。
				この時、メンテナンスフェイズ開始時が適用のタイミングである「自動γ」も解決する。
			\item ターンプレイヤーが優先権を得る。
		\end{enumerate}
	\item[終了時]
		\begin{enumerate}
			\item メンテナンスフェイズ終了時がタイミングの効果を解決する。
				この時、メンテナンスフェイズ終了時が適用タイミングである「自動γ」も解決する。
			\item ドローフェイズへ移行する。
		\end{enumerate}
\end{description}
\subsection{ドローフェイズ}
\begin{description}
	\item[開始時]
		\begin{enumerate}
			\item ターンプレイヤーは 1 ドローする。
				これはドローフェイズの規定の効果である。
			\item ドローフェイズ開始時がタイミングの効果を解決する。
				この時、ドローフェイズ開始時が適用のタイミングである「自動γ」も解決する。
			\item ターンプレイヤーが優先権を得る。
		\end{enumerate}
	\item[終了時]
		\begin{enumerate}
			\item ドローフェイズ終了時がタイミングの効果を解決する。
				この時、ドローフェイズ終了時が適用タイミングである「自動γ」も解決する。
			\item メインフェイズへ移行する。
		\end{enumerate}
\end{description}
\subsection{メインフェイズ}
\begin{description}
	\item[開始時]
		\begin{enumerate}
			\item メインフェイズ開始時がタイミングの効果を解決する。
				この時、メインフェイズ開始時が適用のタイミングである「自動γ」も解決する。
			\item ターンプレイヤーが優先権を得る。
		\end{enumerate}
	\item[終了時]
		\begin{enumerate}
			\item メインフェイズ終了時がタイミングの効果を解決する。
				この時、メインフェイズ終了時が適用タイミングである「自動γ」も解決する。
			\item ディスカードフェイズへ移行する。
		\end{enumerate}
\end{description}
\subsection{ディスカードフェイズ}
\begin{description}
	\item[開始時]
		\begin{enumerate}
			\item ターンプレイヤーの手札が上限枚数を超えていた場合、
				上限枚数になるように手札を選んで破棄する。
				これはディスカードフェイズの規定の効果である。
			\item ディスカードフェイズ開始時がタイミングの効果を解決する。
				この時、ディスカードフェイズ開始時が適用のタイミングである「自動γ」も解決する。
			\item ターンプレイヤーが優先権を得る。
		\end{enumerate}
	\item[終了時]
		\begin{enumerate}
			\item ディスカードフェイズ終了時がタイミングの効果を解決する。
				この時、ディスカードフェイズ終了時が適用タイミングである「自動γ」も解決する。
			\item ターン終了時の処理へ移行する。
		\end{enumerate}
\end{description}
\subsection{ターン終了時}
\begin{enumerate}
	\item 全てのキャラクターに蓄積されているダメージをリセットする。
	\item ターン終了時までの効果の適用を終了する。
		次の行動に影響を及ぼす効果が使用されていない場合もこの時点で終了する。
	\item ターン終了時がタイミングの効果を解決する。
		この時、ターン終了時が適用のタイミングである「自動γ」も解決する。
	\item ターンの間が条件の効果の適用を終了する。
	\item 次のプレイヤーのターンへ移行する。
\end{enumerate}

\end{document}

